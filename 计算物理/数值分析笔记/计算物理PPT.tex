\documentclass[aspectratio=169]{beamer}

\usepackage{ctex} % 中文支持
\usepackage{hyperref}
\usepackage{amsmath} % 导入数学公式宏包
\usepackage{babel}
\usepackage[T1]{fontenc}
\usepackage{latexsym,amsfonts,amssymb,xcolor,multicol,booktabs,calligra}
\usepackage{graphicx,pstricks,listings,stackengine}
\usepackage{caption,subcaption}

% 设置 Beamer 文档的页眉
\setbeamertemplate{headline}
{
  \begin{beamercolorbox}[ht=3ex,dp=1ex,center]{section in head/foot}
    \includegraphics[height=6cm]{pic/scuname.png} % 插入图片
    \quad
    \insertshorttitle % 插入标题
  \end{beamercolorbox}
}
% \setbeamertemplate{footline}
% {
%   \begin{beamercolorbox}[sep=0.5cm,wd=\paperwidth,leftskip=0.5cm,rightskip=0.5cm]{footlinecolor}
%     \includegraphics[height=1cm]{pic/scuname.png}\hfill
%     \insertframenumber/\inserttotalframenumber
%   \end{beamercolorbox}
% }

\usepackage{listings}

% 定义MATLAB代码的颜色样式
\definecolor{mComment}{rgb}{0.133,0.545,0.133} % 绿色
\definecolor{mString}{rgb}{0.627,0.126,0.941} % 紫色
\definecolor{mKeyword}{rgb}{0,0,1} % 蓝色
\definecolor{mBackground}{rgb}{0.9,0.9,0.9} % 背景色

% 设置MATLAB代码的样式
\lstset{
    language=Matlab, % 设置语言为MATLAB
    backgroundcolor=\color{mBackground}, % 设置背景色
    commentstyle=\color{mComment}, % 设置注释的颜色
    keywordstyle=\color{mKeyword}, % 设置关键词的颜色
    stringstyle=\color{mString}, % 设置字符串的颜色
    basicstyle=\ttfamily\footnotesize, % 设置基本样式
    breakatwhitespace=false,         
    breaklines=true,                 
    captionpos=b,                    
    keepspaces=true,                 
    numbers=left,                    
    numbersep=5pt,                  
    showspaces=false,                
    showstringspaces=false,
    showtabs=false,                  
    tabsize=2
}

% Setup TikZ
\usepackage{tikz}
\usetikzlibrary{arrows}

\mode<presentation>
{
  \usetheme{CambridgeUS} % 使用主题
  \setbeamercovered{transparent} % 设置覆盖效果
}

% 设置演示文稿信息
\title[谱方法及其应用]{谱方法及其应用}
\author{李先宝 \and 林子熙 \and 刘子宁 \and 陈睿 \and 段尚福 \and 郑文龙}
\institute[Sichuan University]{Sichuan University}
\date{}
% 在每个 subsection 开始时显示导航条
\AtBeginSubsection[]
{
  \begin{frame}{Outline}
    \tableofcontents[currentsection,currentsubsection]
  \end{frame}
}
\begin{document}

\begin{frame}
    \titlepage
    \begin{figure}
        \centering
        \includegraphics[width=0.2\textwidth]{pic/sculogo.png}
    \end{figure}
\end{frame}


  \section{带电粒子的电磁激发}
\subsection{引言}

\begin{frame}
    \centering
\includegraphics[height=0.5\linewidth,width=0.5\linewidth]{pictures/rabi.png}
\end{frame}
\begin{frame}{引言}
    在这部分中,我们将展示如何解决一维薛定谔方程,描述一个粒子置于势阱中并受到电磁脉冲激发的情况。在位置表象$|\vec{r}>$ 并以\textbf{原子单位制}中,这种情况下的方程可以表示为:
    \begin{equation}
    i\dfrac{\partial\psi}{\partial t}(x,t)=\left[\dfrac{1}{2}\hat{P}^2+V(x)+A(t)\hat{P}\right]\psi(x,t)
   \label{eq:薛定谔方程}
\end{equation}
在这里$\hat{P}$是动量算符,$A(t)$ 是磁矢势,$V(x)$粒子所处的势场。
\pause
其中$A(t)$和$V(x)$满足
\begin{itemize}
    \item $V(x)=-V_0e^{-\alpha x^2}$ 
    \item $A(t)=A_0e^{-\frac{t^2}\sigma}\sin\omega t.$
\end{itemize}
\end{frame}



\begin{frame}{势函数图象}
    \centering
    \includegraphics[height=0.5\linewidth,width=1\linewidth]{picture/_figure_V_0=4.png}

\end{frame}


\subsection{定态薛定谔方程}
\begin{frame}{将定态方程展开成基函数的形式}
要解决这个方程,首先需要找到定态薛定谔方程的本征函数$\phi_n$及其对应的能量$E_n$。
\begin{equation}
    \left[\dfrac{1}{2}\hat{P}^2+V(x)\right]\phi(x)=E\phi(x)
\label{eq:定态薛定谔方程}
\end{equation}
\pause
我们将尝试一个形式的近似解:
\begin{equation}\phi(x)=\sum_{k=0}^Na_k\varphi_k(x)
    \label{eq:定态薛定谔方程近似解}
\end{equation}
    
式中$\varphi_k(x)=c_kH_k(x)e^{-\frac{x^2}2}$ 厄密函数

其中$H_k(x)$是 Hermite 多项式,
$c_k=\frac{1}{\sqrt{2^kk!\sqrt
\pi}}$
\pause
定义实数域上,满足以下正交关系。
\begin{equation}
         \int_{-\infty}^{+\infty}\varphi_k(x)\varphi_{k'}(x)dx=\delta_{kk'}
\label{eq:正交性关系}
        \end{equation}
\end{frame}
\begin{lstlisting}
    % 定义生成 Hermite 函数的函数
function [out] = Hermite(x, N)
% 通过递推生成 N-1 阶 Hermite 函数(注意Hermite函数是从0阶开始的)
H(:, 1) = 1 / sqrt(sqrt(pi)) * exp(-x .^ 2 / 2); % H_0
H(:, 2) = 1 / sqrt(2 * sqrt(pi)) * 2 * x .* exp(-x .^ 2 / 2); % H_1
for k = 3:N
H(:, k) = sqrt(2 / (k - 1)) * x .* H(:, k - 1) - sqrt((k - 1 - 1) / (k - 1)) * H(:, k - 2);
end
out = H; % 返回0到 N-1 阶厄米函数共N个厄密函数
end
\end{lstlisting}
\begin{frame}{求展开系数}

    为了求得展开系数系数$a_k$,我们将近似解代入方程(\ref{eq:定态薛定谔方程})。


    \begin{equation}
       \sum_{k=0}^{N}-\frac{1}{2}\frac{d^{2}}{dx^{2}}a_{k}\varphi_{k}(x)+\sum_{k=0}^{N}V(x)a_{k}\varphi_{k}(x)=E\sum_{k=0}^{N}a_{k}\varphi_{k}(x)
   \end{equation}
   
   然后,我们将其乘以$\varphi_l(x)$并在整个实数域上进行积分。这样我们得到:
   \begin{equation}
        \sum_{k=0}^{N}-\frac{1}{2}a_{k}\int_{-\infty}^{+\infty}\varphi_{l}(x)\frac{d^{2}\varphi_{k}}{dx^{2}}(x)+\sum_{k=0}^{N}a_{k}\int_{-\infty}^{+\infty}\varphi_{l}(x)V(x)\varphi_{k}(x)=E\sum_{k=0}^{N}a_{k}\int_{-\infty}^{+\infty}\varphi_{l}(x)\varphi_{k}(x)
   \label{eq:定态薛定谔方程近似解}
       \end{equation}
       \pause
       因此我们需要三个积分写成矩阵形式
\end{frame}
\begin{frame}{利用厄密函数的性质求解第一个积分矩阵}
    因厄密函数有
    \begin{equation}
        \boxed{ \varphi_{k}^{\prime\prime}(x)=\frac{\sqrt{k(k-1)}}{2}\varphi_{k-2} + \frac{2k+1}{2}\varphi_{k} + \frac{\sqrt{(k+1)(k+2)}}{2}\varphi_{k+2}}
        \label{eq:二阶导递推}
    \end{equation}
代入我们的第一个积分得矩阵元:
\begin{align}
    F_{kl} &= \int_{-\infty}^{+\infty}\varphi_{l}(x)\varphi_{k}^{\prime\prime}(x)dx \notag \\
    &= \int_{-\infty}^{+\infty}\left(\frac{\sqrt{k(k-1)}}{2}\varphi_{l}\varphi_{k-2}+\frac{2k+1}{2}\varphi_{l}\varphi_{k}+\frac{\sqrt{(k+1)(k+2)}}{2}\varphi_{l}\varphi_{k+2}\right)dx \notag \\
    &= \frac{\sqrt{k(k-1)}}{2}\delta_{k-2,l} + \frac{2k+1}{2}\delta_{kl} + \frac{\sqrt{(k+1)(k+2)}}{2}\delta_{k+2,l}
\end{align} 
\end{frame}

\begin{frame}{代入势函数求解第二个积分}
紧接着,我们处理方程(\ref{eq:定态薛定谔方程近似解})式中左边的第二个积分,我们代入势函数。
\begin{equation}
\begin{aligned}
    & \int_{-\infty}^{+\infty}\varphi_{l}(x)V(x)\varphi_{k}(x)dx \\
    & = -V_{0}\int_{-\infty}^{+\infty}\varphi_{l}(x)e^{-\alpha x^{2}}\varphi_{k}(x)dx \\
    & = -\frac{V_{0}}{\sqrt{\alpha}}\int_{-\infty}^{+\infty}\varphi_{l}\left(\frac{y}{\sqrt{\alpha}}\right)e^{-y^{2}}\varphi_{k}\left(\frac{y}{\sqrt{\alpha}}\right)dy \\
    & = \int_{-\infty}^{+\infty}f_{l,k}(y)e^{-y^{2}}dy
\end{aligned}
\end{equation}
这里$f_{k,l}=-\frac{V_{0}}{\sqrt{\alpha}}\varphi_{l}(\frac{y}{\sqrt{\alpha}})\varphi_{k}(\frac{y}{\sqrt{\alpha}}).$
\end{frame}

\begin{frame}{}
    \begin{table}[h]
        \centering
        \begin{tabular}{|c|c|c|c|}
        \hline
        $[a,b]$ & $\omega(x)$ & $\mathbf{Symbol}$ & Name \\ \hline
        $[-1,1]$ & $1$ & $P_n(x)$ & \text{勒让德多项式}\\ \hline
        $[-1,1]$ & $(1-x^{2})^{-1/2}$ & $T_n(x)$ & \text{切比雪夫多项式} \\ \hline
        $[-1,1]$ & $(1-x^{2})^{-1/2}$ & $U_n(x)$ & \text{第二类切比雪夫多项式} \\ \hline
        $[0,\infty)$ & $\exp(-x)$ & $L_n(x)$ & \text{拉盖尔多项式}\\ \hline
        $(-\infty,\infty)$ & $e^{-x^2}$ & $H_n(x)$ & \text{厄米多项式} \\ \hline
        \end{tabular}
        \end{table}
\end{frame}
\begin{frame}{利用高斯型求积公式得到第二个矩阵元}
    所以这里将$e^{-y^2}$视作一个权函数,则其对应的是\textbf{厄密-高斯型求积公式}
    \begin{equation}
        \boxed{ U_{kl}=\int_{-\infty}^{+\infty}f_{l,k}(y)e^{-y^2}dy\approx\sum_{i=0}^{N}\omega_if_{l,k}(y_i)}
        \end{equation}
\end{frame}
\begin{lstlisting}
    % 定义计算 Hermite 多项式的权重和节点的函数
function [out] = GaussHermite(n)
% 利用 Gauss 积分法确定权重和节点
% 构建伴随矩阵 CM
i = 1:n-1;
a = sqrt(i / 2);
CM = diag(a, 1) + diag(a, -1);
% 确定节点 (x) 和权重值 (w)
[V, L] = eig(CM);
[x, ind] = sort(diag(L));
V = V(:, ind)';
w = sqrt(pi) * V(:, 1) .^ 2;
out = [x, w]; % 返回 n 阶厄米多项式的节点对应的权重
end
\end{lstlisting}
\begin{frame}{\textbf{定态薛定谔方程的矩阵形式}}
\begin{equation}
        \boxed{ \left[-\frac{1}{2}F+U\right]\vec{a}=E\vec{a}}
\end{equation}
    这是一个特征值问题,利用MATLAB中的eig函数求得特征值$E_n$和特征向量$\vec{a_n}$。
    \begin{exampleblock}{}
        \pause
    通过设置参数
    \begin{table}[h]
        \centering
        \begin{tabular}{|c|c|c|}
        \hline
        $\mathrm{N}=50$ & $\alpha=1$ & $\mathrm{V}_0=3$ \\
        \hline
        \end{tabular}
        
        得到系统的本征能量$E_n$
        \begin{tabular}{|c|c|c|c|c|c|c|}
        \hline
        $\mathrm{E}_0$ & $\mathrm{E}_1$ & $\mathrm{E}_2$ & $\mathrm{E}_3$ & $\mathrm{E}_4$ & $\mathrm{E}_5$ & $\mathrm{E}_6$ \\
        \hline
        -1.9637 & -0.3699 & 0.0451 & 0.0697 & 0.1847 & 0.2688 & 0.4263 \\
        \hline
        \end{tabular}
        \end{table}
    \end{exampleblock}
\end{frame}
\begin{lstlisting}
    % 定义求解定态本征值问题的函数
function [out] = Pbvaps(N, alpha, V0)
% 求解定态本征值问题
% 构建矩阵 F
F = zeros(N + 1, N + 1);
for i = 1:N+1
if (i > 2)
F(i, i - 2) = sqrt((i - 1) * ((i - 1) - 1)) / 2;
end
F(i, i) = - (2 * (i - 1) + 1) / 2;
if (i < N)
F(i, i + 2) = sqrt(((i - 1) + 1) * ((i - 1) + 2)) / 2;
end
end
% 构建矩阵 U
U = zeros(N + 1, N + 1);
H = GaussHermite(2 * (N + 1) + 1); 
x = H(:, 1); % 横坐标点
w = H(:, 2); % 权重
Phi = Hermite(x / sqrt(alpha), N + 1);
for i = 1:N + 1
for j = 1:N + 1
U(i, j) = -V0 / sqrt(alpha) * dot(w, Phi(:, j) .* Phi(:, i));
end
end
[V, D] = eig(-1 / 2 * F + U); % 求解特征值问题
E = zeros(N + 1, 1);
E = diag(D);
[E, ind] = sort(E); % 对特征值排序
V = V(:, ind);
out = [V, E]; % 返回特征向量和特征值
end

\end{lstlisting}
\begin{lstlisting}
    function out = phi(n, x, a)% 定义计算定态波函数 phi 
% 返回的是 n-1 阶定态波函数(注意波函数的阶数从0阶开始)
v = a(:, n); % 提取 a 的第 n 列
out = zeros(length(x), 1); % 初始化输出
H = Hermite(x, length(v)); % 定义 N 阶厄米函数
for i = 1:length(v)
out = out + v(i) .* H(:, i); % 计算波函数
end
end
\end{lstlisting}
\subsection{含时的薛定谔方程}

\begin{frame}{将波函数展开成基函数的形式}
    确定了系统的本征值,我们就可以求解系统的含时演化。我们将(\ref{eq:薛定谔方程})式写为以下形式:
    \begin{equation}
        i\dfrac{\partial\psi}{\partial t}(x,t)=\left[\hat{H}-iA(t)\dfrac{\partial}{\partial x}\right]\psi(x,t)
    \label{eq:系统的本征方程}
    \end{equation}
    其中$\hat{H}$是系统的哈密顿量,$A(t)$是系统的外场。
    
这里,$\hat{H}=-\frac{\partial^{2}}{\partial x^{2}}+V(x).$为系统的哈密顿算符,$E_n$为其本征值,$A(t)$为系统的外场。
     
设本征态具有形式:
\begin{align}
 \psi_n(x,t) &= b_n(t)\phi_n(x)e^{-iE_nt}
\end{align}

则一般解是这些函数的线性叠加
\end{frame}
\begin{frame}{代入系统的方程}
    我们将$\psi_n(x,t)$代入方程(\ref{eq:薛定谔方程})经过一系列的数学操作得到:
        \begin{equation}
            \boxed{ b'_{m}(t)=\sum_{n=0}^{N}-A(t)b_{n}(t)e^{-i(E_{n}-E_{m})t}\left[\sum_{k=0}^{N}a_{nk}\left(\sqrt{\frac{k}{2}}a_{m,k-1}-\sqrt{\frac{k+1}{2}}a_{l,k+1}\right)\right]}
         \end{equation}
    利用了matlab的ode15s函数可以求解该微分方程
\end{frame}
\begin{lstlisting}
 % 定义计算微分方程中 b 系数的函数
function [G] = g(N, a, A0,t,w,sig, E,b)
I = zeros(N + 1, N + 1);
M = zeros(N + 1, N + 1);
A = A0*sin(w*t)*exp(-t^2/sig);%%定义激励源
for m = 1:N+1
for n = 1:N+1
for k = 1:N+1
if (k == 1)
I(m, n) = I(m, n) - a(n, k) * sqrt((k-1+1) / 2) * a(m, k + 1);
elseif (k == N+1)
I(m, n) = I(m, n) + a(n, k) * sqrt((k-1) / 2) * a(m, k - 1);
else
I(m, n) = I(m, n) + a(n, k) * (sqrt((k-1) / 2) * a(m, k - 1) - sqrt((k-1+1) / 2) * a(m, k + 1));
end
end
M(m,n) = -A*exp(-1i*(E(n,1)-E(m,1))*t) * I(m,n);
end
end
G = M * b; % 计算微分方程中的 b 系数
end
\end{lstlisting}
\subsection{结果展示}
\begin{frame}{结果展示(1)}
    通过设置参数
     \begin{table}[h]
        \centering
        \begin{tabular}{|c|c|c|c|c|}
            \hline
            $\mathrm{A_0=1}$ & $\sigma=20$ & $\mathrm{N=50}$ & $\alpha=1$ & $V_0=3$ \\
            \hline
        \end{tabular}
    \end{table}
    \pause
    \centering
    \includegraphics[height=0.4\linewidth,width=0.5\linewidth]{pictures/_figureA0=1_w=1.5938.png}
\end{frame}
\begin{frame}{结果展示(1)}

    在这种情况下,我们发现2个束缚态,能量分别为$E_0 = -1.9637$和$E_1 = -0.3699$。我们将频率取为前两个束缚态能量的绝对差值,即在这种情况下$\omega = 1.5938$。
    

    我们观察到,一旦脉冲变得显著,粒子处于基态的概率急剧下降,并且它转移到第一个束缚态或连续态的概率增加。
  
    实际上,尽管势能很窄,但并不是很深,因此不需要大量能量来改变粒子的状态。我们看到一旦频率变得可以忽略不计,基态的概率就会增加。这是因为粒子会自发地发射能量量子。
    
\end{frame}
\begin{frame}{结果展示(2)}
    通过设置参数
    \begin{table}[h]
        \centering
        \begin{tabular}{|c|c|c|c|c|}
            \hline
            $\mathrm{A_0=1}$ & $\sigma=20$ & $\mathrm{N=50}$ & $\alpha=0.1$ & $V_0=10$ \\
            \hline
        \end{tabular}
    \end{table}
    \pause
    \begin{center}
        \includegraphics<2->[height=0.4\linewidth,width=0.5\linewidth]{pictures/_figureA0=1_w=2.7275.png}
    \end{center}
\end{frame}
\begin{frame}{结果展示(2)}
    我们发现了13个束缚态!在这种情况下,频率为$\omega= 2.7275$,因此与之前的值相当接近。
我们观察到三个主要差异:
\begin{itemize}
    \item 出现了大量束缚态,
\item 粒子电离的概率很低,
\item 最终概率的分布显示,粒子更有可能结束在第一个束缚态而不是基态。
第二个结果可以解释为势能更深,脉冲太弱以至于不足以使粒子进入连续态。最后一个结果可以解释为现在有更多的束缚态,粒子可以结束在其中,因此留在基态的概率更低。
通过增加激发的幅度,可以增加粒子在其他束缚态中停留一段时间的概率。"
\end{itemize}
\end{frame}
\begin{frame}{拉比振荡}

    拉比振荡是指在量子系统中,当一个外加电磁场以与系统共振的频率作用时,系统中的量子态会发生周期性的振荡。这种振荡通常在原子或分子的能级之间发生,特别是在存在能级间跃迁的情况下。
    \pause
    通过设置参数
    \begin{table}[h]
        \centering
        \begin{tabular}{|c|c|c|c|c|}
            \hline
            $\mathrm{A_0=1}$ & $\sigma=40$ & $\mathrm{N=50}$ & $\alpha=0.01$ & $V_0=60$ \\
            \hline
        \end{tabular}
    \end{table}
\pause
\begin{columns}
    \column{.7\textwidth}
    \begin{center}
    \includegraphics<3->[height=0.4\linewidth,width=0.5\linewidth]{pictures/_figureA0=1_w=25.0725.png}
\end{center}
    \column{.3\textwidth}
    \pause
    在基态和第一束缚态之间确实观察到了拉比振荡。其他束缚态和连续态的概率几乎为零。因此,两个最低能级之间的振荡几乎在0和1之间。
\end{columns}
\end{frame}
\begin{frame}
    \centering
\includegraphics[height=0.5\linewidth,width=0.5\linewidth]{pictures/liang.jpg}
\end{frame}
\begin{lstlisting}
    % 设置扰动振幅和 sigma
A0 = 1;% 扰动振幅,用于描述系统的外部扰动
sig = 40; % sigma,用于描述扰动的宽度
% 修改能量的参数
N = 50; % 物态函数的数量,这是我们要计算的物态的数量
alpha = 0.01; % 物态函数的 alpha 参数,这是势能函数的参数
V0 = 60; % 势能深度,描述势能井的深度





% 设置时间范围和时间点
t_i = -15; % 初始时间,描述我们要模拟的时间范围的开始
t_f = 15; % 最终时间,描述我们要模拟的时间范围的结束
tgrid=1000;%设置格点的数目
tspan = linspace(t_i, t_f, tgrid); % 生成包含 tgrid个时间点的时间范围,这是我们要在其上求解微分方程的时间点


% 计算物态与能量
B = Pbvaps(N, alpha, V0); % 调用 Pbvaps 函数计算物态与能量
E = B(:, N+2); % 提取能量
a = B(:, 1:N+1); % 提取物态(a的每一列指每一个定态本征态的展开系数a)
w = abs(E(1,1)-E(2,1)) ; % Fr閝uence 
% 初始化初始条件
c = zeros(N+1, 1); % 初始化 c,c 是微分方程的初始条件
c(1, 1) = 1; % 设置第一个元素为 1




% 求解微分方程
[t, b] = ode15s(@(t, b) g(N, a, A0,t,w,sig,E, b), tspan, c); % 调用 ode15s 函数求解微分方程
% 纠缠态和连续态计算
b1 = abs(b(:, 1)).^2; % 计算第一个物态的概率
b2 = abs(b(:, 2)).^2; % 计算第二个物态的概率
% 阈值处理
for i = 1:N
if (b1(i, 1) > 1)
b1(i, 1) = 1; % 如果第一个物态的概率大于 1,则将其设置为 1
end
if (b2(i, 1) > 1)
b2(i, 1) = 1; % 如果第二个物态的概率大于 1,则将其设置为 1
end
end
% 判断能量小于 0 的态的数量
p = 0;
for i = 1:N
if (E(i) < 0)
p = p + 1; % 如果能量小于 0,则 p 加 1
end
end
% 计算低于 0 能量态的叠加态
EL = zeros(tgrid, 1);
if (p > 2)
for i = 3:p
EL = EL + abs(b(:, i)).^2; % 计算低于 0 能量态的叠加态
end
end
% 计算连续态
CO = 1 - abs(b(:, 1)).^2 - abs(b(:, 2)).^2 - EL; % 计算连续态
% 处理连续态中小于 0 的情况
for i = 1:100
if (CO(i, 1) < 0)
CO(i, 1) = 0; % 如果连续态小于 0,则将其设置为 0
end
end
%绘制概率随时间的变化图
figure(1)
subplot(2,1,1)
plot(t, b1, 'color', '#FFBE7A', 'LineWidth', 2);
hold on;
plot(t, b2, 'color', '#FA7F6F', 'LineWidth', 2);
plot(t, EL, 'color', 'g', 'LineWidth', 2);
plot(t, CO, 'color', '#2878b5', 'LineWidth', 2);
legend('基态', '第一束缚态', '其他束缚态', '连续态', 'Location', 'best') % 添加图例,放在最佳位置
xlabel('时间 (单位)') % 添加 x 轴标签
ylabel('概率') % 添加 y 轴标签
title(sprintf('概率随时间的变化, A_0=%.2f, ω=%.2f, V_0=%.2f', A0, w, V0));
grid on; % 添加网格线

% 计算激发图
t0 = linspace(t_i, t_f, tgrid); % 生成包含 tgrid个时间点的时间范围
A = A0 * sin(w .* t0).* exp(-t0 .^ 2 / sig); % 计算激发
subplot(2,1,2)
plot(t0, A, 'm-') % 绘制激发图,使用品红色线条
title('激发') % 添加标题
xlabel('时间 (单位)') % 添加 x 轴标签
ylabel('A(t)') % 添加 y 轴标签
grid on; % 添加网格线
hold off
fileName1 = ['_figureA0=', num2str(A0), '_w=', num2str(w), '.png'];
saveas(figure(1), fullfile("C:\latex\ 模版\pictures", fileName1));

% figure(2)
% %绘制势能图
% x = linspace(-25, 25, 3000); % 生成包含 300 个位置点的位置范围
% x = x';
% V = -V0 * exp(-alpha * x .^ 2); % 计算势能
% plot(x, V, 'b-') % 绘制势能图,使用蓝色线条
% title('势能') % 添加标题
% xlabel('位置 (单位)') % 添加 x 轴标签
% ylabel('V(x)') % 添加 y 轴标签
% grid on; % 添加网格线
% fileName2 = ['_figure', '_V_0=', num2str(V0), '.png'];
% saveas(figure(2), fullfile("C:\Users\40147\Desktop\pictures", fileName2));


% %定义文件夹路径
% folderPath ="C:\Users\aheic\Desktop\计算物理";
% 
% % 定义文件名
% filename = sprintf('wave_animation_A0=%.2f_w=%.2f', A0, w);
% 
% % 创建视频对象
% writerObj = VideoWriter(folderPath, 'MPEG-4');
% open(writerObj);
% % 创建动画
% 
% for i = 1:length(t)
%     bt = b(i, :);
%     psi1 = 0; % 定义连续态
%     psi2 = 0; % 定义束缚态
%     for j = p+1:N+1
%         psi1 = psi1 + bt(j) * phi(j, x, a) .* exp(-1i * E(j) * t(i));
%     end
%     psi2 = psi2 + (b1(i) * phi(1, x, a) .* exp(-1i * E(1) * t(i))) + (b2(i) * phi(2, x, a) .* exp(-1i * E(2) * t(i)));
% 
%     psi1 = abs(psi1).^2;
%     psi2 = abs(psi2).^2;
% 
%     subplot(3,1,1)
%     plot(x, psi1)
%     title(sprintf('连续态波函数 (t=%.2f, A_0=%.2f, ω=%.2f, V_0=%.2f)', t(i), A0, w, V0));
%     xlabel('位置(单位)')
%     ylabel('概率')
%     grid on
%     subplot(3,1,2)
%     plot(x, psi2)
%     xlim([-25, 25]);
%     ylim([0, 1]);
%     title(sprintf('束缚态波函数 (t=%.2f, A_0=%.2f, ω=%.2f, V_0=%.2f)', t(i), A0, w, V0));
%     grid on
%     subplot(3,1,3)
%     plot(t(1:i),A(1:i))
%      xlabel('时间 (单位)') % 添加 x 轴标签
%      ylabel('A(t)') % 添加 y 轴标签
%     xlim([-15,15])
%     ylim([-A0,A0])
%     title('激发') % 添加标题
%     pause(0.000002) % 调整暂停时间,以适应你的可视化速度
% 
%     % 将当前帧添加到视频中
%     frame = getframe(gcf);
%     writeVideo(writerObj, frame);
% end
% 
% % 关闭视频文件
% close(writerObj);


\end{lstlisting}


































































































































































































































































































































































































































































































































































































































\end{document}
